% !TEX TS-program = pdflatex
% !TEX encoding = UTF-8 Unicode

% This is a simple template for a LaTeX document using the "article" class.
% See "book", "report", "letter" for other types of document.

\documentclass[11pt]{article} % use larger type; default would be 10pt

\usepackage[utf8]{inputenc} % set input encoding (not needed with XeLaTeX)


%%% PAGE DIMENSIONS
\usepackage{geometry} % to change the page dimensions
\geometry{a4paper} % or letterpaper (US) or a5paper or....


\usepackage{graphicx} % support the \includegraphics command and options

% \usepackage[parfill]{parskip} % Activate to begin paragraphs with an empty line rather than an indent

%%% PACKAGES
\usepackage{booktabs} % for much better looking tables
\usepackage{array} % for better arrays (eg matrices) in maths
\usepackage{paralist} % very flexible & customisable lists (eg. enumerate/itemize, etc.)
\usepackage{verbatim} % adds environment for commenting out blocks of text & for better verbatim
\usepackage{subfig} % make it possible to include more than one captioned figure/table in a single float
\usepackage{cite}
% These packages are all incorporated in the memoir class to one degree or another...

%%% HEADERS & FOOTERS
\usepackage{fancyhdr} % This should be set AFTER setting up the page geometry
\pagestyle{fancy} % options: empty , plain , fancy
\renewcommand{\headrulewidth}{0pt} % customise the layout...
\lhead{}\chead{}\rhead{}
\lfoot{}\cfoot{\thepage}\rfoot{}

%%% SECTION TITLE APPEARANCE
\usepackage{sectsty}
\allsectionsfont{\sffamily\mdseries\upshape} % (See the fntguide.pdf for font help)
% (This matches ConTeXt defaults)

%%% ToC (table of contents) APPEARANCE
\usepackage[nottoc,notlof,notlot]{tocbibind} % Put the bibliography in the ToC
\usepackage[titles,subfigure]{tocloft} % Alter the style of the Table of Contents
\renewcommand{\cftsecfont}{\rmfamily\mdseries\upshape}
\renewcommand{\cftsecpagefont}{\rmfamily\mdseries\upshape} % No bold!

\usepackage{hyperref}


\title{VesiclePy: A Python API for Vesicle Detection}
\author{Team-Awesome: Mary Yen, Zhou Li, Jizhou Xu}
%\date{} % Activate to display a given date or no date (if empty),
         % otherwise the current date is printed 

\begin{document}
\maketitle

\section{Introduction}

In order to diagnose and treat neurological disorders, we need a mapping of the human brain. Although there are large volumes of brain image data available, there are still challenges in creating the mapping. The main mapping structures, the synapses, are difficult to detect due to their small size and poor image contrast. VESICLE is the first program that detects mammalian synapses with consideration to biological contexts. With VESICLE, large volumes of image data may be processed, but VESICLE is currently written in MATLAB, which relies on too many dependencies. This problem can be solved if VESICLE is written in Python. Python offers more open source capabilities, and Python itself offers efficient open-source libraries for use.
%Description of what your project is. Perhaps start off with an overview of the area of interest \textbackslash [and cite it, of course]. It is good practice to use the OCAR \cite{noel2013} technique which we will discuss in class: after you introduce your topic explain why this work is difficult, what you wish to do to overcome this, and why it has yet to be done. Finally, summarize why your progress on this project will be meaningful. This shouldn't be much longer than 5 or 6 sentences.

\section{Project Outline}
\subsection{Methods}
We plan to choose from existing python libraries such as SciPy and NumPy for the actual random forest algorithm. If time permits, depends on how well our API performs, we might tweak the implementation and test out additional features that might affect the detection accuracy.
\subsection{Schedule}
Jan.7th - 
\begin{itemize}
	\item Read Matlab tutorial (\url{http://docs.neurodata.io/vesicle/tutorials/vesiclerf.html}).
	\item Understand MATLAB VESICLE (document/take note of the MATHLAB code).
\end{itemize}
Jan.16th - 
\begin{itemize}
	\item Finish up Coding.
	\item	Test python libraries we can use in place of tool folder files.
	\item Minimize dependency, refactor code, etc.
\end{itemize}
Jan 22nd -
\begin{itemize}
	\item Add additional features to the data points and try to improve performance.
	\item Finalize report.
	\item Finalize poster.
\end{itemize}

\subsection{Allocation of Tasks}
Zhou:
\begin{itemize}
	\item Documenting MATLAB VESICLE.
	\item Poster Making.
	\item Coding.
\end{itemize}
Jizhou:
\begin{itemize}
	\item Coding.
	\item Finding alternate dependencies.
	\item Refactoring.
\end{itemize}
Mary:
\begin{itemize}
	\item Coding.
	\item Documenting Python code.
	\item Testing.
\end{itemize}
\bibliography{yourbibname}{}
\bibliographystyle{plain}

\end{document}