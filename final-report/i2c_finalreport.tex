% !TEX TS-program = pdflatex
% !TEX encoding = UTF-8 Unicode

% This is a simple template for a LaTeX document using the "article" class.
% See "book", "report", "letter" for other types of document.

\documentclass[11pt]{article} % use larger type; default would be 10pt

\usepackage[utf8]{inputenc} % set input encoding (not needed with XeLaTeX)


%%% PAGE DIMENSIONS
\usepackage{geometry} % to change the page dimensions
\geometry{a4paper} % or letterpaper (US) or a5paper or....


\usepackage{graphicx} % support the \includegraphics command and options

% \usepackage[parfill]{parskip} % Activate to begin paragraphs with an empty line rather than an indent

%%% PACKAGES
\usepackage{booktabs} % for much better looking tables
\usepackage{array} % for better arrays (eg matrices) in maths
\usepackage{paralist} % very flexible & customisable lists (eg. enumerate/itemize, etc.)
\usepackage{verbatim} % adds environment for commenting out blocks of text & for better verbatim
\usepackage{subfig} % make it possible to include more than one captioned figure/table in a single float
\usepackage{cite}
% These packages are all incorporated in the memoir class to one degree or another...

%%% HEADERS & FOOTERS
\usepackage{fancyhdr} % This should be set AFTER setting up the page geometry
\pagestyle{fancy} % options: empty , plain , fancy
\renewcommand{\headrulewidth}{0pt} % customise the layout...
\lhead{}\chead{}\rhead{}
\lfoot{}\cfoot{\thepage}\rfoot{}

%%% SECTION TITLE APPEARANCE
\usepackage{sectsty}
\allsectionsfont{\sffamily\mdseries\upshape} % (See the fntguide.pdf for font help)
% (This matches ConTeXt defaults)

%%% ToC (table of contents) APPEARANCE
\usepackage[nottoc,notlof,notlot]{tocbibind} % Put the bibliography in the ToC
\usepackage[titles,subfigure]{tocloft} % Alter the style of the Table of Contents
\renewcommand{\cftsecfont}{\rmfamily\mdseries\upshape}
\renewcommand{\cftsecpagefont}{\rmfamily\mdseries\upshape} % No bold!



\title{Mapping the Brain: An Introduction to Connectomics\\vesiclePy}
\author{Zhou Li, Jizhou Xu, Mary Yen}
%\date{} % Activate to display a given date or no date (if empty),
         % otherwise the current date is printed 

\begin{document}
\maketitle

\section{Abstract}

We have a program capable of mapping the human brain by vesicle clustering available in Matlab. However, this code has many Matlab dependencies and isn't as open-sourced and fast as Python code. Therefore, converting this code into Python code is viable. There are some challenges in that Matlab doesn't translate directly into Python due to differences in data format and dependencies, but we still were able to find a basic pipeline that is able to perform the basic detection actions by using the Python scikit library, RandomForestClassifier to aid in the classifying process of the program. For future work, we would like to do a more complete conversion of vesiclePy. Due to the large size of the program and our team's infamiliarity with Matlab, doing a basic pipeline was the most realistic gaol for a three week course. With more time, we would like to properly learn the algorithms that the Matlab code uses and attempt to improve the code if possible using Python libraries.

\section{Results}

Here you should detail what came from your project (i.e. what data do you have to show for your work). Here you should include a qualitative and numerical description of the results themselves, as well as at least 1 \emph{good} figure. A \emph{good} figure means one that displays data both clearly, effectively, and aesthetically. This can be a difficult thing to do, so if you have questions don't be afraid to ask us. \\

This whole final report should be no longer than 1 page of writing, and no more than 2 full pages with the figure(s) (include them as an appendix). Of course, don't forget references (those of you using LaTeX can grab the proposal template for how to use insert the references again).\\

Good luck!

The current code takes data that 

\end{document}

























